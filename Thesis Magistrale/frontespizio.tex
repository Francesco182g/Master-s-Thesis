\linespread{1} % per il frontespizio utilizzo l'interlinea singola

\thispagestyle{empty}
\large

% % % % % % % % % % % % % % % % % % % % 
% TESTO IN CORSIVO
% % \emph{\textbf{UNIVERSITA' DEGLI STUDI DI Salerno}}\\
% % \textbf{\emph{Corso di Laurea in Informatica Magistrale}}\\
% % \textbf{\emph{Corso di Statistica e Analisi dei Dati}}
% % % % % % % % % % % % % % % % % % 

%INTESTAZIONE DEL FRONTESPIZIO
\begin{center}
	\LARGE{\uppercase{Università degli Studi di Salerno}}\\
	\vspace{2mm}
\end{center}

\begin{center}   
	\LARGE{{\textsc{Dipartimento di Informatica}}}
	\\    
	
\end{center}

%******************************************************************
%                                   Logo UniSa
%******************************************************************
\begin{figure}[h]
	\begin{center}
		\includegraphics[scale=0.37]{figure/logo_standard.png}
		% nella cartella ``Immagini'' ci sono diversi loghi da poter scegliere
	\end{center}
\end{figure}
%******************************************************************                                

\vspace{0.2cm}

%
%
%TITOLO DELLA TESI
\begin{center}
	{\LARGE{TESI DI LAUREA MAGISTRALE}}\\
	\vspace{1cm}
	{\LARGE \textbf{An Augmented Reality Mobile Application for Skin Lesion Data Visualization} \smallskip\\}                                               
	%   {\LARGE \myTitle}
\end{center}


\vspace{15mm}
\noindent
%
\begin{minipage}[t]{0.47\textwidth}
	%Relatore
	{\large{ Relatori:\\\bf Prof.ssa Rita Francese}}\\
	%Correlatore
	{\large{ \bf Prof. Michele Risi}}\\
	{\large{ \bf Dott.ssa Maria Frasca}}
	
\end{minipage}
\hfill
\begin{minipage}[t]{0.4\textwidth}\raggedleft
	%Candidato
	{\large{Candidato: \\ \bf Francesco Garofalo\\ Mat. 0522500615}}
\end{minipage}                                     					              

%ANNO ACCADEMICO     
\vspace{1cm}
\begin{center}
	\textsc{Anno Accademico 2019/2020}
\end{center}

%Nuova pagina    
\newpage

