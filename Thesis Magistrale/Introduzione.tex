\chapter*{Introduzione}
\addcontentsline{toc}{chapter}{Introduzione}
\lhead{\bfseries }
\rhead{\thepage}
Il melanoma cutaneo è un tipo di cancro molto aggressivo, la cui incidenza è in aumento in tutto il mondo.
\newline
La sua diagnosi può essere effettuata utilizzando una tecnica di imaging chiamata \textbf{dermoscopia}, che amplifica la lesione e consente l'osservazione delle strutture sottocutanee. Tuttavia, l'uso di questa tecnica richiede un lungo addestramento, la diagnosi è spesso soggettiva e difficile da riprodurre. Per questi motivi è necessario sviluppare metodi diagnostici automatici.
\newline
Il lavoro svolto in questa tesi mira a sviluppare metodi di analisi delle immagini e riconoscimento di modelli per diagnosticare immagini dermoscopiche;
\newline
In particolare, questo lavoro è diviso in due fasi.
\newline
Nella prima fase è stata addestrata una rete neurale convoluzionale (CNN) che si occupa dell'analisi dei nevi al fine di stabilire se il nevo in questione è un melanoma o meno con un valore compreso tra 0 e 1 che indica l'accuratezza dell'analisi.
La rete è stata addestrata con il dataset HAM 10000, che offre 10000 immagini dermatoscopiche (catturate mediate un dermatoscopio digitale) in diversi scenari e con immagini contenti nevi di diverso tipo:
\begin{itemize}
\item 6705 nevi melanociti
\item 1099 lesioni benigne
\item 1113 melanomi
\end{itemize}
Per bilanciare i dati è stata applicata una tecnica di data aumentation sulle immagini dei melanomi.
\newline

In seguito sono state analizzate diverse metodologie ed algoritmi destinati all'estrazione delle caratteristiche del nevo.

In primo luogo il frame, contenente il nevo, viene sottoposto ad analisi e a preprocessing al fine di garantire la qualità ottimale prima di passare all'estrazione delle caratteristiche e alla classificazione.
\newline
Il preprocessing assicura che l'immagine sia ritagliata correttamente, priva di peli e ben segmentata;
Dopodiché a partire dal frame di un nevo preprocessato vengono estratte le feature, ovvero: asimmetria, irregolarità dei bordi, colore, dimensione frattale, distanza del centroide, e la visione del nevo in 2D photometric stereo.\footnote{Visione del nevo in 3D rapportata in una immagine 2D.}
\newline
La seconda parte è stata dedicata allo sviluppo implementativo dell'applicazione in realtà aumentata, il cui nome è Naevus\footnote{Vedi Appendice A}.
\newline
L'applicazione utilizza un paradigma client-server con il framework Ionic/Angular per il Client e con Django/Python per il Server; in questo modo è stato possibile affidare al server i calcoli più complessi, la cui precisione è fondamentale al fine di ottenere risultati robusti e soddisfacenti, mentre l'applicazione mobile (lato client) comunica con il server e si occupa di mostrare i dati in AR all'utente.
\newline
L'interfaccia dell'applicazione mobile è semplice ed intuitiva in modo da rendere l'utilizzo da parte del medico/dematologo privo di difficoltà.
\newline
Infine sono stati valutati diversi scenari di utilizzo dell'applicazione in diversi ambienti, e diverse condizioni di luce in modo da valutare la bontà degli algoritmi definiti ed implementati nell'applicazione finale.
\newline
Questo lavoro di tesi è articolato come segue:
\begin{itemize}
\item Nel primo capitolo è stato descritto brevemente il background degli argomenti trattati, in modo da fornire al lettore una overview semplificata e sintetica necessaria a comprendere gli argomenti trattati;
\item Nel secondo capitolo vengono affrontati e descritti i sistemi e le ricerche esistenti nell'ambito dell'analisi dei nevi, l'utilizzo di applicazioni mobile in medicina e le reti neurali in ambito medico;
\item Nel terzo capitolo è stato descritto il dataset utilizzato per l'addestramento della rete neurale convoluzionale;
\item Nel quarto capitolo è stata definita la rete neurale convoluzionale utilizzata per l'analisi dei nevi, partendo dal processo di preprocessing delle immagini fino ad arrivare all'addestramento vero e proprio;
\item Nel quinto capitolo sono state descritte le metodologie utilizzate per l'analisi del nevo ovvero il preprocessing delle immagini, l'estrazione delle feature ABCD, l'utilizzo del classificatore creato, e lo sviluppo di una metodologia per rendere l'applicazione compatibile con la realtà aumentata;
\item Nel sesto capitolo è stata descritto il design e l'implementazione dell'applicazione finale e le scelte che sono state compiute in fase di progettazione ed in corso d'opera;
\item Nel settimo capitolo è stata descritta l'applicazione creata con diversi scenari di funzionamento;
\item ed infine sono state stilate le conclusioni sul lavoro di tesi svolto e gli eventuali progetti e sviluppi futuri del lavoro. 
\end{itemize}
Buona Lettura.