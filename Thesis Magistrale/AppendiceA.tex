\appendix
\chapter{Appendice A - Naevus}
\label{app:nomeAppendiceA}

La scelta del nome dell'applicazione "Naevus" deriva dal latino "nævus" sostantivo della seconda declinazione maschile singolare, il cui significato è appunto "neo", "macchia naturale della pelle" oppure "voglia", in alcuni contesti può significare (in senso figurato) anche "difetto" o "macchia".
Il nome è stato preso da una celebre frase latina il cui autore è sconosciuto:
\newline
\textit{Naevus pulchram faciem non deformat},
\newline
"Un neo non guasta un bel viso".